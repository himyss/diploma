\usepackage[left=2.5cm,right=1.5cm,
top=2cm,bottom=2cm,bindingoffset=0cm]{geometry} % вроде бы поля

\usepackage[utf8]{inputenc} % размер
\textwidth  160mm
\textheight 230mm
\topmargin -24pt
\oddsidemargin -5mm
\evensidemargin-5mm
\setlength{\parindent}{1cm}

\usepackage{setspace}  % межстрочный интервал
\usepackage{amsmath}
% полуторный интервал
\onehalfspacing
\renewcommand{\rmdefault}{ftm} % Times New Roman

\title{Diploma}
\author{IvanM}
\date{april 2017}

\usepackage[T2A]{fontenc} % указывает внутреннюю кодировку TeX

\usepackage[english,russian]{babel}   %% загружает пакет многоязыковой вёрстки
%\usepackage{fontspec}      %% подготавливает загрузку шрифтов Open Type, True Type и др.
%\defaultfontfeatures{Ligatures={TeX},Renderer=Basic}  %% свойства шрифтов по умолчанию
%\setmainfont[Ligatures={TeX,Historic}]{Times New Roman} %% задаёт основной шрифт документа
%\setsansfont{Comic Sans MS}                    %% задаёт шрифт без засечек
%\setmonofont{Courier New}
\usepackage{indentfirst}
%\frenchspacing

%\usepackage[intlimits]{amsmath}

%%% Работа с картинками
\usepackage[labelformat=simple]{subcaption}
% метка subfigure: "(а)" вместо дефолтного "а"
\renewcommand\thesubfigure{(\alph{subfigure})} 
\usepackage{array,graphicx,caption}
%\usepackage{graphicx}  % Для вставки рисунков
\usepackage{color}		%для наверно цветов
%\usepackage{wrapfig} % Обтекание рисунков текстом

\renewcommand{\epsilon}{\ensuremath{\varepsilon}}
\renewcommand{\phi}{\ensuremath{\varphi}}
\renewcommand{\kappa}{\ensuremath{\varkappa}}
\renewcommand{\le}{\ensuremath{\leqslant}}
\renewcommand{\leq}{\ensuremath{\leqslant}}
\renewcommand{\ge}{\ensuremath{\geqslant}}
\renewcommand{\geq}{\ensuremath{\geqslant}}
\renewcommand{\emptyset}{\varnothing}

\newcommand{\red}[1]{\textcolor{red}{#1}}
\newcommand{\er}{\textmd{EXPERTroot}}
