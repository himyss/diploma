\section{Полученные результаты}

%\subsection{Временные тесты прототипы NeuRad}
\label{NeuRadres}
%После обработки результатов с эксперимента с прототипов детектора NeuRad в Дубне, нам удалось оценить достаточное количество параметров данного детектора.
%Типичные формы сигналов полученных с прототипа можно увидеть на рис.~\ref{ris:signals}.

%\begin{figure}[h]
%	\center{\includegraphics[width=1\linewidth]{signals.png}}
%	\caption{Сигналы, полученные с одной стороны прототипа NeuRad.}
%	\label{ris:signals}
%\end{figure}
%{
%	\centering
%	\includegraphics[width=1\linewidth]{signals.png}
%	\captionof{figure}{Сигналы, полученные с одной стороны прототипа NeuRad.}\label{ris:signals}
%}

Форма полученных сигналов не была похожа на теоретически ожидаемую (см. Рис.\ref{ris:signal1}). Это можно объяснить множеством физических эффектов, таких, как: фотоны внутри волокна могут испытывать многократное отражение от его границ; в результате отражений фотоны могут попасть на пиксель, несоответствующий волокну, в котором они "родились"; несмотря на все нанесённые светоизолирующие покрытия на поверхности каждого волокна, остаётся вероятность прохождения фотонов из одного волокна в другое.

%Несмотря на достаточно необычную форму сигналов, суммарный вид сигнала, принял более привычную форму: рис.~\ref{ris:integralform}, что позволило нам рассчитать время высвечивания сцинтиллятора (метод расчёта описан в главе \ref{section:processing}).

Несмотря на достаточно необычную форму сигналов, вероятнее всего сигналы состоят из совокупности так называемых одноэлектронных сигналов. Одноэлектронным сигналом будем называть сигнал, возникающий на аноде ФЭУ в результате выбивания одного электрона с фотокатода. Поэтому суммарный вид сигналов принял привычную форму, из которой было определено, что время высвечивания равно 5,8\,нс, для данного сцинтиллятора которое равняется 3,2\,нс \cite{crystals}.

\begin{figure}[h]
	\center{\includegraphics[width=1\linewidth]{integralform.png}}
	\caption{Суммарный сигнал с одного канала.}
	\label{ris:integralform}
\end{figure}
%{
%	\centering
%	\includegraphics[width=1\linewidth]{integralform.png}
%	\captionof{figure}{Суммарный сигнал с одного канала.}\label{ris:integralform}
%}
%Таким образом, время высвечивания, оцененное методами \er, составило \red{число}.
%Время нарастания сигнала с 10 до 90\% 

Очевидно, величина $\Delta\tau$ зависит от множества случайных процессов, таких, как: продольная координата взаимодействия падающей частицы с материалом детектора, энергия электрона отдачи, количество испущенных сцинтиллятором фотонов, достигнувших фотокатода ФЭУ и других. Форма распределения этой величина ожидалась схожей формой нормального распределения случайной величины. Поэтому временное разрешение прототипа детектора NeuRad оценивалось как параметр полной ширины на половине высоты (англ. FWHM - full width at half maximum) распределения разницы времён сигналов ($\Delta\tau$) с противоположных сторон одного оптического волокна.

Параметр FWHM вычислялся по формуле:
\begin{equation}
\label{FWHM}
{\rm FWHM}=2\sqrt{2\ln(2)}\sigma,
\end{equation}
где $\sigma$ - есть стандартное отклонение, которое рассчитывалось фитированием формы распределения $\tau$ функцией Гаусса\cite{vratislav}. 

Параметры методов расчёта времени сигналов подбирались таким образом, чтобы FWHM распределения $\Delta\tau$ было наименьшим. 

На Рис.\ref{ris:Tau} показаны распределения $\Delta\tau$, рассчитанное с помощью методов описанных в главе\,\ref{section:processing}. Хотя пучок был коллимирован посередине продольной длины детектора, математическое ожидание (Mean) всех распределений принимает значение около 10\,нс. Это объясняется тем, что кабеля для снятия сигналов с ФЭУ были разной длины, и разница времени задержки сигналов была равна 10\,нс. 

%Распределение этой величины можно увидеть на Рис.\ref{ris:TauLEDCFD}. Медиана таких распределений близка к нулю, что было ожидаемо, в силу коллимации источника посередине прототипа. 

%\begin{figure}[!h]
%	\centering
%	\includegraphics[width=0.8\linewidth]{tau.png}
%	\caption{Распределение времён сигналов с противоположных сторон одного оптического волокна.}\label{ris:Tau}
%\end{figure}

{
	\centering
	\includegraphics[width=0.8\linewidth]{tau.png}
	\captionof{figure}{Распределение $\Delta\tau$ для сигналов с противоположных сторон одного оптического волокна.}\label{ris:Tau}
}


Слева вверху (Constant Fraction Discriminator): распределение $\Delta\tau$, разницы времён сигналов с противоположных сторон одного оптического волокна, рассчитанное методом CFD с параметрами: время задержки 1.5 нс, коэффициент ослабления инвертированного сигнала  0.3.

Справа вверху (Leading Edge Analysis): $\Delta\tau$ рассчитано методом LED с порогом 10\,mV. 

Внизу : $\Delta\tau$ рассчитано методом анализа переднего фронта сигнала с порогом 10 mV. Время сигнала рассчитывали, как момент достижения 50\%(слева) и 10\%(справа) амплитуды первого локального максимума.
 
{
	\centering
	\begin{tabular}{|c|c|c|}
		\hline
		\textbf{Метод}: & Параметры & $\sigma$/FWHM[нс]\\
		\hline
		CFD: & Задержка  1,5\,нс, коэффициент ослабления 0,3 & 1,28/3,02\\
		\hline
		LED: & Порог 20\,мВ &  1,86/4,39\\
		\hline
		Анализ фронта 50\%: & Порог 20\,мВ &  1,33/3,13\\
		\hline
		Анализ фронта 10\%: & Порог 20\,мВ &  1,16/2,72\\
		\hline
	\end{tabular}
	\captionof{table}{ Результаты рассчётов временного разрешения прототипа NeuRad.}\label{tab:results}
}
 
%Оценка амплитудный характеристик также немаловажную роль при анализе результатов.  Очевидно, что чем больше максимальная амплитуда сигнала, тем больше заряд, зарегистрированный анодом ФЭУ. То, что эта зависимость линейна было также проверено. 
 
Временное разрешение, полученное из распределений на Рис.\ref{ris:Tau} не удовлетворило ожидания. Поэтому были сделан анализ отбора событий и корректировки сигналов. 

Была замечена сильная зависимость $\Delta\tau$ для метода CFD от суммарного заряда, регистрируемых ФЭУ с двух сторон оптического волокна, см. Рис.\ref{ris:walk}а). Очевидно, что такая зависимость является следствием не физических явлений, а свойством записи данных.

{
	\centering
	\includegraphics[width=\linewidth]{walkcorr.png}
	\captionof{figure}{а) Зависимость $\Delta\tau$, оцененное методом CFD от заряда и распределние $\Delta\tau$ до корректировки. b) Данные после корректировки.}\label{ris:walk}
}


%\begin{figure}[!h]
%	\centering
%	\includegraphics[width=\linewidth]{walkcorr.png}
%	\caption{а) Зависимость $\tau$, оцененное методом CFD от суммарного заряда и распределние $\tau$ до корректировки. b) Данные после корректировки.}\label{ris:walk}
%\end{figure}

Поэтому, полученный параметр $\Delta\tau$ был скорректирован так, чтобы максимально уменьшить визуально очевидную зависимость $\Delta\tau$ от заряда. Для этого, каждое рассчитанное $\Delta\tau$ изменялось следующим образом:

\begin{equation}
\Delta\tau' = \Delta\tau + \frac{k}{Q},
\end{equation}
где $\Delta\tau'$ - новое, скорректированное значение разницы времён, $Q$ - заряд, $k$ - коэффициент корректировки, подбирающийся так, чтобы в результате корректировки минимизировать зависимость разницы времён сигналов от заряда. В корректировке, показанной на Рис.\ref{ris:walk}, $k$ был равен 2\,нс.

На Рис.\ref{ris:walk}b) показана зависимость $\Delta\tau'$ от заряда, получившаяся в результате корректировки и распределение $\Delta\tau'$. 

Вторым методом уменьшения значения временного разрешения был отбор событий. Как описывалось выше, наиболее интересными являются сигналы с большой максимальной амплитудой и, соответственно, большим регистрируемым зарядом и длинными по времени. Наиболее удачным параметром для отбора был выбран параметр Time-over-Threshold(ToT), который рассчитывался, как разница времён пересечения сигнала уровня, равного половине максимальной амплитуды сигнала, заднего и переднего фронтов сигнала.

На Рис.\ref{ris:totcorr} показано распределение параметра ToT для двух рассматриваемых каналов, $\Delta\tau$ после отбора. Условие отбора заключается в том, что для рассматриваемых событий ToT должен быть больше 3, но меньше 8\,нс.

\begin{figure}[!h]
	\centering
	\includegraphics[width=0.8\linewidth]{totcorr.png}
	\caption{а) Распределение ToT для двух каналов одного оптического волокна. b) Распределение разницы времён сигналов для событий, удовлетворяющих условию отбора. }\label{ris:totcorr}
\end{figure}

Таким образом, лучшая оценка временного разрешения детектора была рассчитана методом анализа переднего фронта сигнала с привязкой к точке 10\% амплитуды от максимальной и равно 2,59\,нс. Такой временная неопределённости детектирования падающей частицы соответствует неопределённости по координате 47\,см, см. формулу\,\ref{eq:distance}. 

\subsection{Эксперимент с пластиной сцинтиллятора в GSI}

%Как и ожидалось, сигналы на тонкой пластине сцинтиллятора, оказались более плавильной формы, см. рис.\ref{ris:GSIsignal}.

%\begin{figure}[h]
%	\center{\includegraphics[width=1\linewidth]{GSIsignals.png}}
%	\caption{Сигналы, полученные с осциллографа Le Croy, в эксперименте с тонкой пластиной сцинтиллятора.}
%	\label{ris:GSIsignal}
%\end{figure}
%{
%	\centering
%	\includegraphics[width=1\linewidth]{GSIsignals.png}
%	\captionof{figure}{Сигналы, полученные с осциллографа Le Croy, в эксперименте с тонкой пластиной сцинтиллятора.}\label{ris:GSIsignal}
%}

Все описанные методы в главе\,\ref{NeuRadres} были применены для данных, собранных на установке на территории GSI.

%\begin{figure}[h]
%	\centering
%\center{\includegraphics[width=1\linewidth]{deltaT1.png}}
%	\caption{Распределение разницы времён сигналов с противоположных сторон пластицы сцинтиллятора, полученное с помощью методов, заложенных в осциллографе Le Croy.}
%	\label{ris:deltaT}
%\end{figure

Полученные результаты описывающие временные характеристики изображены на Рис.\ref{ris:GSIcfd_amp}. Как и ожидалось, временное разрешение такой установки лучше, чем разрешение прототипа NeuRad, так как в данном случае отсутствует вклад пространственного разрешения из-за малого размера чувствительной части детектора.

В верхней части рисунка\,\ref{ris:GSIcfd_amp} изображены распределения разницы времён сигналов, рассчитанное методом диксриминатора со следящим порогом до отбора по параметру ToT слева и после отбора справа. Данный отбор являлся наиболее удачным для набранных данных, о чём свидетельствует нижняя часть рисунка. В нижней части изображена зависимость суммы максимальных амплитуд сигналов в двух каналов от $\Delta\tau$. Визуально очевидно, что данный отбор отсеивает подавляющее большинство событий, c отклонённым значением $\Delta\tau$ от ожидаемой.

%\begin{figure}[h]
%\center{\includegraphics[width=1\linewidth]{CFD_amp.png}}
%	\caption{Корреляция разницы времён сигналов, оцененных методом CFD, с суммой амплитуд сигналов с эксперимента с пластиной сцинтиллятора. Справа - эта же корреляция, но с отбором событий, для которых время сигнала над порогом больше 3нс.}
%	\label{ris:GSIcfd_amp}
%\end{figure}
{
	\centering
	\includegraphics[width=1\linewidth]{CFD_amp.png}
	\captionof{figure}{Корреляция разницы времён сигналов, оцененных методом CFD, с суммой амплитуд сигналов с эксперимента с пластиной сцинтиллятора. Справа - эта же корреляция, но с отбором событий, для которых время сигнала над порогом больше 3нс.}\label{ris:GSIcfd_amp}
}

%Наиболее удачным для сохраненных данных оказался метод подсчёта 

%Коэффициент наклона для переднего фронта, один из параметров фитирования, является одним из важных параметров отбора событий. Очевидно. что чем быстрее нарастает сигнал, тем более точно можно определить время сигнала в целом.