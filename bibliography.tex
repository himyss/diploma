\newpage
\begin{thebibliography}{999}
	
	\bibitem{ufn}
	Л.В. Григоренко, М.С. Головков, С.А. Крупко, С.И. Сидорчук, Г.М. Тер-Акопьян, А.С. Фомичев, В. Худоба: ``Исследования легких экзотических ядер вблизи границы стабильности в Лаборатории ядерных реакций ОИЯИ'', УСПЕХИ ФИЗИЧЕСКИХ НАУК, Том 186, №4, с.337-386, 2016.
	
	\bibitem{FAIR}
	M. Winkler, H. Geissel, H. Weick, B. Achenbach, K. Behr, D. Boutin, et al.:
	``The status of the Super-FRS in-flight facility at FAIR'', p. 266, 2008.
	
	\bibitem{inet}
	Б.С. Ишханов, Э.И. Кэбин. Экзотические ядра. М.: Изд. Московского университета, 2002.
	
	\bibitem{book1}
	Б.С.Ишханов, И.М.Капитонов, Н.П.Юдин. Частицы и атомные ядра. -М., Издательство Московского университета, 2005.
	
	\bibitem{FLNR}
	FLNR He10. 2008.
	
	\bibitem{GSI}
	F. Farinon.: Unambiguous Identification and Investigation of Uranium Projectile Fragments. Discovery of 63 New Neutron-rich Isotopes in the Element Range 61<Z<78 at the FRS, Justus-Liebig Universitat Giessen (2011).
	
	\bibitem{crystals}
	Saint-Gobain is a global leader in the manufacture and development of engineered materials such as glass, insulation, reinforcements, containers, building materials, ceramics and plastics.
	The formation of the Crystals Division reflects Saint-Gobain’s commitment to the development of high performance materials. 
	http://www.crystals.saint-gobain.com/	
	
	\bibitem{identificate}
	F. Farinon: ``Unambiguous Identification and Investigation of Uranium Projectile Fragments. Discovery of 63 New Neutron-rich Isotopes in the Element Range 61  Z  78 at the FRS'', Justus-Liebig Universitat Giessen, 2011.
	
	\bibitem{Super-FRS}
	O. Kiselev et al.: ``Si detectors for Time of Flight measurements at the Super-FRS'', GSI Scientic Report 2012, p. 172, 2012.
	
	\bibitem{Bycl}
	O. Kiselev at al.: ``Radiation hardness tests of Si detectors for Time of Flight measurements at the Super-FRS'', GSI Scientific Report 2014, p.137, 2015.
		
	\bibitem{Hamamatsu}
	Hamamatsu Photonics K.K.: Si detectors for high energy particles (Chapter 09). In: OPTO-SEMICONDUCTOR HANDBOOK. S. 226 (6.1 Particle collision experiments. 29 May 2013).
	
	\bibitem{methods}
	Beuzekom, M. (2006). "Identifying fast hadrons with silicon detectors", University of Groningen Faculty of Mathematics and Natural Sciences Dissertation
	
	\bibitem{plata}
	https://www.psi.ch/drs/evaluation-board
	
	\bibitem{tektronix}
	http://ru.tek.com/
	
	\bibitem{er}
	http://er.jinr.ru
	
	\bibitem{dasha}
	D.A. Kostyleva, Diploma thesis,
	"Investigation of the time characteristics of semiconductor detectors"
	Department of nuclear physics
	State University «Dubna», Dubna, Russia, 2016.
	
	\bibitem{vratislav}
	W.~R.~Leo,
	%``Techniques for Nuclear and Particle Physics Experiments: A How to Approach,''
	Berlin, Germany: Springer (1987) 368 p
	%33 citations counted in INSPIRE as of 08 May 2017

	\bibitem{FAIRroot}
	https://fairroot.gsi.de/
	
	\bibitem{geant}
	http://geant4.cern.ch/

	\bibitem{root}
	https://root.cern.ch/

	\bibitem{CBMroot}
	M. Al-Turany, D. Bertini and I. Koenig
	"CbmRoot: Simulation and Analysis framework for CBM Experiment"
	GSI Darmstadt, Germany
	
	\bibitem{PANDAroot}
	https://panda.gsi.de/oldwww/framework/computing.php

	\bibitem{CFDwalk}	
	Heilbronn, Lawrence; Iwata, Yoshiyuki, Iwase, H. Off-line correction for excessive constant-fraction-discriminator walk in neutron time-of-flight experiments, article, October 15, 2003; Berkeley, California. (digital.library.unt.edu/ark:/67531/metadc780778/: accessed May 17, 2017), University of North Texas Libraries, Digital Library, digital.library.unt.edu; crediting UNT Libraries Government Documents Department.
	
	
	\bibitem{vitalik}
	В.Н. Щетинин, С.Г. Белогуров,
	Расчётно-пояснительная записка к дипломному проекту
 	"Имитационное моделирование эксперимента EXPERT 
	физики радиоактивных пучков"
	Москва, 2016.
	
\end{thebibliography}