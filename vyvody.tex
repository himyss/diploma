\section{Выводы}


В работе были разработаны алгоритмы для обработки сигналов ФЭУ, которые были успешно внедрены в уже существующий фреймворк \er. С их помощью было рассчитано временное разрешение прототипа разрабатываемого детектора NeuRad. 

Результаты расчётов были проверены двумя способами: было проведено вспомогательное измерение, было проведено моделирование эксперимента при помощи инструментов \er. В обоих случаях было установлено, что полученные результаты были получены с хорошей точностью.

Обработка оцифрованных сигналов с помощью \er\ обладает большей огромной функциональностью в силу имеющейся возможности вносить корректировки в алгоритмы обработки и добавлять новые. Полученную программу планируются применять для отладки методики экспресс-анализа данных с целью оперативной калибровки сцинтилляционных детекторов в широком энергетическом диапазоне.

%
%Разработка и апробация Монте-Карло кода для описания световыхода сцинтилляционного детектора на основе CsI(Tl)/ФЭУ при регистрации лёгких заряженных частиц и гамма-квантов.
%
%Отладка методики экспресс-анализа данных с целью оперативной калибровки сцинтилляционных детекторов в широком энергетическом диапазоне. 
%
%Результатом работы должна быть отлаженная программа, учитывающая геометрию сцинтиллятора, светосбор, зависимость световыхода от сорта ионизирующего излучения и энергии; программный продукт нацелен на создание базы данных для массива из 64-х сцинтилляционных детекторов.