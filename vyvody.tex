\section{Выводы}

Представляемая дипломная работа посвящена изучению свойств детектора нейтронов NeuRad. В ее рамках были проведены тестовые измерения имеющегося прототипа и наряду изучены свойства нескольких фотоэлектронных умножителей, которых применение планируется при конструкции детектора.
%
Измерения проводились с использованием источников гамма излучения $^{137}$Cs и $^{60}$Co.
%Набранные экспериментальные данные последовательно обрабатывались с помощью созданных алгоритмов для обработки электронных сигналов проторипа.
В рамках работы были разработаны алгоритмы для обработки электронных сигналов, которые обычно применяются в приборах ядерной электроники. 
%Основным  преимуществом их применения является возможность обработки полученных сигналов оффлайн и изменения параметров обработки.
Важной задачей являлась моделирование физических процессов, происходящих в детекторе, и получение данных, структурно тождественных экспериментальным.
%
Все разработанные алгоритмы были успешно внедрены в существующий программный пакет \er, предназначенный для моделирования и обработки данных с экспериментов в рамках коллаборации Super-FRS Experiment.

Не менее важным являлось сравнение данных с эксперимента и моделирования. 
Была найдена подходящая аналитическая форма для моделирования одноэлектронного импульса, для того, чтобы обеспечить форму электронного сигнала соответствующую экспериментальной. Было получено хорошее сходство между сравниваемыми данными.

Обработка интегрального временного сигнала, полученного на осциллографе говорит о том, что значение времени высвечивания равно 5,8\,нс, что соответствует результату, полученному при моделировании (6\,нс). Поскольку при моделировании эксперимента время высвечивания сцинтиллятора BCF12 закладывалось таким, каким оно было предоставлено производителем (3,2\,нс), можно сделать заключение, что форма электронного сигнала искажена другими процессами, и время высвечивания материала соответствует 3,2\,нс.

Тем не менее, наблюдались некоторые отличия от ожидаемых значений. Например, световыход по данным производителя должен быть 8000\,фотонов/МэВ, но для полного соответствия моделирования с экспериментом мы были вынуждены предположить что световыход равен 2000\,фотонов/МэВ, что противоречит паспортным данным сцинтиллятора. Полученное значение может указывать на нестандартный светосбор из волокон прототипа, на который влияет необычная форма детектора, размер и обработка поверхности сцинтилляционных волокон.

Также надо учесть факт, что детектор предназначен для регистрации высокоэнергетических нейтронов посредством регистрации протонов отдачи. От таких частиц можно ожидать большие энергопотери чем от электронов "выбитых" Комптон-эффектом фотонами с энергией меньше чем один 1\,МэВ, как это было в тестовых измерениях. Планируется продолжить текущие тесты, используя меченые нейтроны.

Временное разрешение прототипа было изучено разными методами обработки, причем лучший результат (2,6\,нс) был получен с помощью метода анализа переднего фронта с отбором по времени сигнала над порогом. В то же время тест с тонким сцинтиллятором показал значение временного разрешения 1,7\,нс. Этот факт показывает, что на временное разрешение сильно влияет конструкция пучка сцинтилляционных волокон и крепление к фотоэлектронному умножителю. Предполагается, что другое покрытие поверхности волокон сможет ограничить внутреннее отражения света в одном оптическом волокне и таким образом уменьшить неопределённость определения времени взаимодействия налетающей частицы с материалом детектора.

Обработка оцифрованных сигналов с помощью \er\ обладает огромной функциональностью в силу имеющейся возможности вносить корректировки в алгоритмы обработки и добавлять новые. Полученную программу планируется применять для отладки методики экспресс-анализа данных с целью оперативной калибровки сцинтилляционных детекторов в широком энергетическом диапазоне.

%
%Разработка и апробация Монте-Карло кода для описания световыхода сцинтилляционного детектора на основе CsI(Tl)/ФЭУ при регистрации лёгких заряженных частиц и гамма-квантов.
%
%Отладка методики экспресс-анализа данных с целью оперативной калибровки сцинтилляционных детекторов в широком энергетическом диапазоне. 
%
%Результатом работы должна быть отлаженная программа, учитывающая геометрию сцинтиллятора, светосбор, зависимость световыхода от сорта ионизирующего излучения и энергии; программный продукт нацелен на создание базы данных для массива из 64-х сцинтилляционных детекторов.