\section{Выводы}

Представляемая дипломная работа посвящена изучению свойств детектора нейтронов NeuRad. В ее рамках были проведены тестовые измерения имеющегося прототипа и наряду изучены свойства нескольких фотоэлектронных умножителей, которых применение планируется при конструкции детектора.
%
Измерения проводились с использованием источников гамма излучения $^{137}$Cs и $^{60}$Co.
%Набранные экспериментальные данные последовательно обрабатывались с помощью созданных алгоритмов для обработки электронных сигналов проторипа.
В рамках работы были разработаны алгоритмы для обработки электронных сигналов, которые стандартно применяются в приборах ядерной электроники. Основным преимуществом является возможность обработки полученных сигналов оффлайн и любая настройка их параметров.
Не менее важной задачей являлась симуляция физических процессов в детекторе и получение информации со структурой тождественной экспериментальным данным.
%
Все разработанные алгоритмы были успешно внедрены в существующий фреймворк \er, предназначенный для симмуляций и обработки данных экспериментов в рамках коллаборации SuperFRS Experiment.

Одной из важнейших задач было сравнение экспериментальных данных с симуляцией. 
Мы нашли подходящую аналитическую форму для моделирования одноэлектронного импульса, чтобы обеспечить формирование симулированного электронного сигнала. Было получено хорошее сходство между экспериментальными и симулированными данными.

Обработка интегрального временного сигнала полученного на осциллографе показывает значение времени высвечивания как 5,8\,нс, что очень хорошо соответствует результату симуляции (6\,нс). Поскольку в модель оцифровки симулированных данных мы закладывали время высвечивания сцинтиллятора BCF12 предоставленное производителем (3,2\,нс), можем сделать заключение, что форма электронного сигнала искажена другими процессами и время высвечивания материала соответствует 3,2\,нс.

Тем не менее, мы наблюли несколько отличий от ожидаемых значений. Например световыход по данным производителя должен быть 8000\,фотонов/МэВ, но для полного соответствия симуляции с экспериментов мы были вынуждены предположить 2000\,фотонов/МэВ, что противоречит паспортным данным. Нами полученное значение может показывать на нестандартный светосбор из волокон прототипа, на который влияет необычная форма детектора, размер волокон (особо отношение продольного и поперечного размера) и обработка поверхности сцинтиляционных волокон.

Также надо учесть факт, что детектор предназначен для регистрации высокоэнергетических нейтронов посредством рассеянных протонов. От таких частиц можно ожидать на много больше энергопотери чем от электронов выбитых комптон-эффектом фотонами с энергией меньше чем один МэВ, как это было в наших тестовых измерениях. Планируется продолжить текущие тесты с помощью меченных нейтронов.

Временное разрешение прототипа было изучено с помощью разных методов обработки, причем лучшего результата 2,6\,нс полученного с помощью анализа переднего фронта с отбором по времени над порогом. В то же время тест с тонким сцинтиллятором показал значение 1,7\,нс. Этот факт показывает, что на временное разрешение сильно влияет конструкция пучка сцинтилляционных волокон и крепление к фотоэлектронному умножителю. Предполагаем, что другое покрытие поверхности волокон сможет ограничить внутреннее отражения света в одном файбере и таким образом улучшить определение время рождения фотонов в материале детектора.

Обработка оцифрованных сигналов с помощью \er\ обладает огромной функциональностью в силу имеющейся возможности вносить корректировки в алгоритмы обработки и добавлять новые. Полученную программу планируется применять для отладки методики экспресс-анализа данных с целью оперативной калибровки сцинтилляционных детекторов в широком энергетическом диапазоне.

%
%Разработка и апробация Монте-Карло кода для описания световыхода сцинтилляционного детектора на основе CsI(Tl)/ФЭУ при регистрации лёгких заряженных частиц и гамма-квантов.
%
%Отладка методики экспресс-анализа данных с целью оперативной калибровки сцинтилляционных детекторов в широком энергетическом диапазоне. 
%
%Результатом работы должна быть отлаженная программа, учитывающая геометрию сцинтиллятора, светосбор, зависимость световыхода от сорта ионизирующего излучения и энергии; программный продукт нацелен на создание базы данных для массива из 64-х сцинтилляционных детекторов.