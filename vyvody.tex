\section{Выводы}

Представляемая дипломная работа посвящена изучению свойств детектора нейтронов NeuRad. В ее рамках были проведены тестовые измерения имеющегось прототипа и отдельно изучены свойства нескольких фотоэлектронных умножителей, которых применение планируется при конструкции детектора.

Измерения проводились с использованием источников гамма излучения $^{137}$Cs и $^{60}$Co. Набранные экспериментальные данные последовательно обрабатывались с помощью созданных алгоритмов для обработки электронных сигналов проторипа. 

Абзац про симуляции.


Все разработанные алгоритмы были успешно внедрены в существующий фреймворк \er, предназначенный для симмуляций и обработки данных экспериментов в рамках коллаборации SuperFRS Experiment.

Важнейшей задачей было сравнение экспериментальных данных с симуляцией.

Нашли подходящую аналитическую форму для одноэлектронного импульса \ldots, что важно для формирования сигнала.

Эксперимент и симуляция показывают время высвечивания нашего материала как ХХ\,нс, что противоречит паспортным данным. На полученное значение может сильно влиять необычная форма детектора и обработка сцинтиляционных волокон.

Кроме фиговых параметров текущего прототипа у нас были хреновые сигналы из-за слабенького комптона. Надо сделать что-нибудь с нейтронами или частицами с ионизирующей способностью как-то получше.

\red{В представляемой дипломной работе были разработаны алгоритмы для обработки сигналов ФЭУ, которые были успешно внедрены в уже существующий фреймворк \er. С их помощью было рассчитано временное разрешение прототипа разрабатываемого детектора NeuRad.} 

\red{Результаты расчётов были проверены двумя способами: было проведено вспомогательное измерение, было проведено моделирование эксперимента при помощи инструментов \er. В обоих случаях было установлено, что полученные результаты были получены с хорошей точностью.}

\red{Обработка оцифрованных сигналов с помощью \er\ обладает большей огромной функциональностью в силу имеющейся возможности вносить корректировки в алгоритмы обработки и добавлять новые. Полученную программу планируются применять для отладки методики экспресс-анализа данных с целью оперативной калибровки сцинтилляционных детекторов в широком энергетическом диапазоне.}

%
%Разработка и апробация Монте-Карло кода для описания световыхода сцинтилляционного детектора на основе CsI(Tl)/ФЭУ при регистрации лёгких заряженных частиц и гамма-квантов.
%
%Отладка методики экспресс-анализа данных с целью оперативной калибровки сцинтилляционных детекторов в широком энергетическом диапазоне. 
%
%Результатом работы должна быть отлаженная программа, учитывающая геометрию сцинтиллятора, светосбор, зависимость световыхода от сорта ионизирующего излучения и энергии; программный продукт нацелен на создание базы данных для массива из 64-х сцинтилляционных детекторов.